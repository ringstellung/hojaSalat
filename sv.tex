\documentclass[12pt]{article}

% aprovechamiento de la p\'agina -- fill an A4 (210mm x 297mm) page

% Note: 1 inch = 25.4 mm = 72.27 pt
% 1 pt = 3.5 mm (approx)

% vertical page layout -- one inch margin top and bottom

\topmargin    -18 mm  % top margin less 1 inch (sevilla)
\headheight     0 mm  % height of box containing the head
\headsep        0 mm  % space between the head and the body of the page
\textheight   276 mm
\footskip       7 mm  % distance from bottom of body to bott om of foot

% horizontal page layout -- one inch margin each side
\oddsidemargin    0 mm   % inner margin less one inch on odd pages
\evensidemargin   0 mm   % inner margin less one inch on even pages
\textwidth      159.2 mm % normal width of text on page

\usepackage{arabtex}
\usepackage{atrans}
\usepackage{nashbf}
\usepackage{twoblks}
\usepackage[utf8]{inputenc}
\usepackage[absolute]{textpos}
\usepackage{graphicx}

\usepackage{tikz,everypage}
\usetikzlibrary{backgrounds}
\newcommand*{\AbsolutePosition}[3]{%
    % #1 = x (from south west corner of page)
    % #2 = y
    % #3 = content
    \AddThispageHook{%
        \begin{tikzpicture}[remember picture,overlay]
            \draw (current page.south west) ++ (#1,#2) node[opacity=0.7] {#3};%
        \end{tikzpicture}
    }
}

%%%%%%%%%%%%%%%%%%%%%%%%%%%%%%%%%%%%%%%%%%%%%%%%
%% cómo generar el código QR
%% compilar con:  pdflatex -shell-escape <file>
%% la primera vez para generar el pdf y luego 
%% comentar.
%%%%%%%%%%%%%%%%%%%%%%%%%%%%%%%%%%%%%%%%%%%%%%%%
% \usepackage{pst-barcode}
% \usepackage{auto-pst-pdf}   
%%%%%%%%%%%%%%%%%%%%%%%%%%%%%%%%%%%%%%%%%%%%%%%%%

%% selección de la fuente global
\usepackage[
            light,
            math]
           {kurier}
\usepackage[T1]{fontenc}

\newenvironment{parrafo}{\begin{trivlist}\item []}{\end{trivlist}}
\pagestyle{empty}
\tracingarab=0
\setarab
\settrans{spanish}
\novocalize
\arabtrue

\pagestyle{empty}

\begin{document}

% %\begin{textblock*}{290mm}(185mm,258mm)
% \begin{textblock*}{290mm}(25mm,2mm) % Granada y Motril
% %\begin{textblock*}{290mm}(25mm,4mm) % Sevilla
%   \includegraphics[scale=1]{taqwim.pdf}
% \end{textblock*}

\begin{center}
\large\textsc{Fundación Mezquita de Sevilla}
\end{center}
\AbsolutePosition{0mm}{40mm}{\includegraphics[scale=0.7]{bolas.jpg}}

\footnotesize \twoblocks{Parte de los momentos del
  \arabfalse\transtrue \RL{.salAT} correspondiente al mes de
  \arabfalse\transtrue\settransfont{\bf\itshape} 
  \RL{rama.dAn} \rm 
  del a\~no \textbf{1437 h.} (2015/2016 d.J.) para la
  ciudad de \textbf{Sevilla} y cercan\'{\i}as.  }{
  \begin{arabtext}
    \noindent
    .hi.s.saTu 'awqAti a.s-.salATi li^sahri \setnashbf 
    rama.dAn
    \setnash min `Ami \setnashbf <\bf 1437> h- . \setnash(
    \LR{2015}/\LR{2016} m- .) limadInaTi \setnashbf 'i^sbIlIyaT
    \setnash wa-mA ^gAwarahA.
  \end{arabtext}}

\normalsize

\begin{parrafo}
\centerline{
\begin{tabular}{|c|c|c|c|c|c|c|c|c|c|}
\hline
\RL{al-`i^sA'}&
\RL{al-ma.grib}&
\RL{al-`a.sr}&\RL{a.z-.zuhr}&
\RL{al-^surUq}&\RL{al-fa^gr}&
\multicolumn{2}{|c|}{\RL{yawmu al-'usbU`}}&
\RL{yUliyuh} /\RL{yUniyuh}&
\RL{rama.dAn}\\
\hline
\arabfalse\transtrue \RL{`i^sA'} &
\arabfalse\transtrue \RL{ma.grib}&
\arabfalse\transtrue \RL{`a.sr}  &
\arabfalse\transtrue \RL{.zuhr}  &
\arabfalse\transtrue \RL{^surUq} &
\arabfalse\transtrue \RL{fa^gr}  &
                      \multicolumn{2}{|c|}{d\'{\i}a semana}&
                      jun./jul.  &
\settransfont{\bf\itshape}\arabfalse\transtrue
\RL{rama.dAn} \rm\\
\hline
23:35&21:50&18:15&14:28&06:56&05:11&mar&\RL{a_t-_tulA_tA'u}&07*&01\\\hline
23:35&21:50&18:16&14:28&06:56&05:11&mie&\RL{al-'arbi`A'u}  &08*&02\\\hline
23:36&21:51&18:16&14:28&06:55&05:10&jue&\RL{al-_hamIsu}    &09*&03\\\hline
23:37&21:52&18:16&14:28&06:55&05:10&vie&\RL{al-^gumu`aTu}  &10*&04\\\hline
23:37&21:52&18:16&14:29&06:55&05:10&sab&\RL{as-sabtu}      &11*&05\\\hline
23:38&21:52&18:17&14:29&06:55&05:09&dom&\RL{al-'a.hadu}    &12*&06\\\hline
23:39&21:53&18:17&14:29&06:55&05:09&lun&\RL{al-'i_tnaynu}  &13*&07\\\hline
23:39&21:53&18:17&14:29&06:55&05:09&mar&\RL{a_t-_tulA_tA'u}&14*&08\\\hline
23:40&21:54&18:17&14:29&06:55&05:09&mie&\RL{al-'arbi`A'u}  &15*&09\\\hline
23:40&21:54&18:18&14:30&06:55&05:09&jue&\RL{al-_hamIsu}    &16*&10\\\hline
23:41&21:54&18:18&14:30&06:55&05:09&vie&\RL{al-^gumu`aTu}  &17*&11\\\hline
23:41&21:55&18:18&14:30&06:55&05:09&sab&\RL{as-sabtu}      &18*&12\\\hline
23:41&21:55&18:18&14:30&06:56&05:09&dom&\RL{al-'a.hadu}    &19*&13\\\hline
23:42&21:55&18:18&14:31&06:56&05:09&lun&\RL{al-'i_tnaynu}  &20*&14\\\hline
23:42&21:55&18:19&14:31&06:56&05:10&mar&\RL{a_t-_tulA_tA'u}&21*&15\\\hline
23:42&21:56&18:19&14:31&06:56&05:10&mie&\RL{al-'arbi`A'u}  &22*&16\\\hline
23:42&21:56&18:19&14:31&06:56&05:10&jue&\RL{al-_hamIsu}    &23*&17\\\hline
23:42&21:56&18:19&14:31&06:57&05:10&vie&\RL{al-^gumu`aTu}  &24*&18\\\hline
23:42&21:56&18:20&14:32&06:57&05:11&sab&\RL{as-sabtu}      &25*&19\\\hline
23:42&21:56&18:20&14:32&06:57&05:11&dom&\RL{al-'a.hadu}    &26*&20\\\hline
23:42&21:56&18:20&14:32&06:58&05:12&lun&\RL{al-'i_tnaynu}  &27*&21\\\hline
23:42&21:56&18:20&14:32&06:58&05:12&mar&\RL{a_t-_tulA_tA'u}&28*&22\\\hline
23:42&21:56&18:20&14:32&06:59&05:13&mie&\RL{al-'arbi`A'u}  &29*&23\\\hline
23:42&21:56&18:20&14:33&06:59&05:13&jue&\RL{al-_hamIsu}    &30*&24\\\hline
23:42&21:56&18:21&14:33&06:59&05:14&vie&\RL{al-^gumu`aTu}  &01*&25\\\hline
23:41&21:56&18:21&14:33&07:00&05:15&sab&\RL{as-sabtu}      &02*&26\\\hline
23:41&21:56&18:21&14:33&07:00&05:15&dom&\RL{al-'a.hadu}    &03*&27\\\hline
23:41&21:56&18:21&14:33&07:01&05:16&lun&\RL{al-'i_tnaynu}  &04*&28\\\hline
23:40&21:56&18:21&14:34&07:01&05:17&mar&\RL{a_t-_tulA_tA'u}&05*&29\\\hline
23:40&21:55&18:21&14:34&07:02&05:18&mie&\RL{al-'arbi`A'u}  &06*&30\\\hline
\end{tabular}
}

% Lugares de interés
                                         
% Güera
% s-m 1h 21m 22s
% m-a 1d 07h 57m
% Cape Town
% s-m 1h 00m 15s
% m-a 1d 05h 08m

% informe mes anterior:
                                        
\centerline{                             
\begin{tabular}{|c|c|c|c|c|c|}           
  \hline                                   
  \multicolumn{6}{|c|}{Algunos par\'ametros lunares en \textbf{Sevilla} para
    final de \arabfalse\transtrue\settransfont{\bf\itshape} 
  \RL{rama.dAn} \rm}\\
  \hline
  d\'{\i}a de observar&
  puesta de Sol&
  puesta de Luna&
  dif. Luna-Sol&
  edad lunar&
  ?`avistamiento?\\
  \hline
  05/07/2016&
  21h 50m 14s&
  22h 35m 10s&
  00h 44m 56s&
  01d 08h 49m&
  $\bullet\bullet\bullet\bullet\circ$\\ 
  \hline
\end{tabular}
}
 \vspace{2mm}

 \footnotesize \noindent \textbf{Sevilla} (37º23'N/5º59'O): longitud
 de referencia 15ºE, zona horaria +1 h, altitud sobre el nivel del mar
 150 m, \'angulo de crep\'usculo 18º, \arabfalse\transtrue \RL{qiblaT}
 100.4º, declinaci\'on --1.69.
\end{parrafo}

 \scriptsize
 \begin{parrafo}
   \begin{center}
     Plaza Ponce de León 9, 41003 Sevilla. España --- 
     Tel: +34 954 022 979 --- Fax: +34 954 048 747\\
     \texttt{info{\@}mezquitadesevilla.com} ---
     \texttt{www.mezquitadesevilla.com}
   \end{center}
 \end{parrafo}

%  \begin{parrafo}
%  \centerline{\texttt{Segunda Edición}}
%  \end{parrafo}
\end{document}
 